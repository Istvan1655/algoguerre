\chapter{Etude de l'existant}

	\markboth{\MakeUppercase{Etude de l'existant}}{}
	
	\section{Intro}

	Originellement le jeu de la guerre (KriegSpiel en Allemand) était un système d'entraînement utilisé par le Royaume de Prusse pour la formation de ses officiers. Créé en 1812 par Georg von Rassewitz, sous officier de l'armée prussienne, puis perfectionné par la suite par son fils, il fut des années plus tard l'inspiration de Guy Debord, cinéaste et homme de lettres français, pour créer sa propre version en 1977.\\	
	
	Dans sa forme actuelle, le KriegSpiel est à la fois un jeu de type Wargame sur lequel furent basés d'autres jeux par la suite, tout comme une variante du jeu d'échec classique.\\

	Le KriegSpiel est un jeu de plateau faisant s'opposer deux joueurs contrôlant chacun une armée.\\

	Une version logicielle permettant de jouer en local et en réseau fut développée par le Radical Software Group (RSG), un groupement de développeurs et d'artistes. La mise en ligne de cette version numérique ne fut pas bien accueillit par les héritiers de la pensées et des droits des créations de Debord qui réclamèrent le retrait de cette version. La veuve de M. Debord ayant déclaré son envie de ne pas voir la création de son mari sur internet.
		
	\clearpage
